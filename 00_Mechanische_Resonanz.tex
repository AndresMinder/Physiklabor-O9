%Dokumentklasse
\documentclass[a4paper,11pt]{scrreprt}

% ============= Packages =============
% Standard Packages
\usepackage[utf8]{inputenc}
\usepackage[ngerman]{babel}
\usepackage[T1]{fontenc}
\usepackage{graphicx}
\usepackage{fancyhdr}
\usepackage{blindtext}
\usepackage{lmodern}
\usepackage{color}
\usepackage{hyperref}
\usepackage{url}
\usepackage{wrapfig}
\usepackage{ulem}
\usepackage[bottom=3.5cm, top=2cm, left=2cm, right=2cm]{geometry}
\usepackage{tabularx}
\usepackage{booktabs}
\usepackage{microtype}
\usepackage{subfigure}
\usepackage{float}
\usepackage{framed,color}


% zusätzliche Schriftzeichen der American Mathematical Society
\usepackage{amssymb}
\usepackage{amsfonts}
\usepackage{amsmath}

% Literaturverzeichnisstil
\bibliographystyle{unsrt}

\parindent0pt

% ============= Kopf- und Fußzeile =============
\pagestyle{fancy}
%
\fancyhead[L]{\includegraphics[scale=1]{Bilder/FHNW.jpg}}
\fancyhead[C]{EIT}
\fancyhead[R]{\slshape Physiklabor}
%%
\fancyfoot[L]{}
\fancyfoot[C]{}
\fancyfoot[R]{}
%
\renewcommand{\headrulewidth}{0.4pt}
\renewcommand{\footrulewidth}{0pt}

% ============= Dokumentbeginn =============

\begin{document}
%Seiten ohne Kopf- und Fußzeile sowie Seitenzahl

\begin{tabular}{p{\textwidth}}

	\begin{center}
		%\includegraphics[scale=0.5]{img/logos.jpg}
	\end{center}

\\

	\begin{center}
		\Huge{\textsc{\textbf{Laborheft}}}
	\end{center}

\vspace*{2cm}

	\begin{flushleft}
		\begin{tabular}{lll}
			\LARGE \textbf{Versuchsleiter}: & 	\hspace{2cm} & \LARGE Andres Minder\\
			\LARGE \textbf{Assistent}: 		& 	\hspace{2cm} & \LARGE Nando Spiegel\\
		\end{tabular}
	\end{flushleft}

\vspace*{2cm}

	\begin{center}
	\renewcommand{\arraystretch}{4}
	\large
		\begin{tabular}{|l|l|l|l|}
			\hline 
			\textbf{Durchführung} & \textbf{Versuch} & \textbf{Abgabe} & \textbf{Akzeptiert} \\ 
			\hline 
			06.03.2018 & W6 - Mech. Resonanz mit Fahrbahnpendel & 20.03.2018 &  \\ 
			\hline 
			17.04.2018 & O9 - Interferenz und Beugung 			& 01.05.2018 &  \\ 
			\hline 
			29.05.2018 & A11 - Röntgenstrahlung / -beugung 		& 12.06.2018 &  \\ 
			\hline 
		\end{tabular}
	\end{center}
		 

\end{tabular}

\include{00b_titelblatt}
 
\tableofcontents \thispagestyle{fancy} \cfoot{} \renewcommand{\footrulewidth}{0pt} 

% ab hier ist diese Konvention für die Fusszeile eingestellt
\fancyfoot[L]{glaL4}
\fancyfoot[C]{\thepage}
\fancyfoot[R]{\today}
\renewcommand{\footrulewidth}{0.4pt}

\chapter{Arbeitsgrundlagen}
% ==================================================================
\setcounter{page}{1} \thispagestyle{fancy} 
% ==================================================================
Um die theoretische Quintessenz dieser Arbeit kurz zu erläutern, werden hier die \textit{\nameref{subsec:fresnelBeobachtungsart}} und die \textit{\nameref{subsec:frauenhofBeobachtungsart}} gezeigt. Dabei wird die \textit{\nameref{subsec:beugspalt}}, \textit{\nameref{subsec:beugloch}} und die \textit{\nameref{subsec:beugstrichgitter}} betrachtet.

\section{Theoretische Grundlagen}
Beugung ist die Abweichung von der geradlinigen Wellenausbreitung, mit Hilfe von Strahlen wird sie beschrieben. Wenn Wellen auf Oberflächen mit Begrenzungen, wie in dieser Arbeit z.B. der Spalt treffen, dann tritt die Beugung auf.\\[0.5cm]
Erklärt werden können diese Beugungseffekte mit Hilfe des Huygens-Fresnel’schen Prinzips. Dabei senden alle Punkte hinter einer Öffnung Sekundär-Kugelwellen aus, deren Überlagerung das neue Wellenfeld liefert. Durch diese Interferenz der Kugelwellen resultiert so mit der Fresnel'schen oder der Frauenhofer'schen Beobachtungsart ein Muster aus Licht (Maximas) und Schatten (Minimas) auf dem Schirm. Dies wird Interferenzmuster genannt\cite{Angaben2011}.\\

\subsection{Fresnel'sche Beobachtungsart}
\label{subsec:fresnelBeobachtungsart}
Um das Interferenzmuster direkt beobachten zu können wird hier ein Schirm in das Nahfeld gebracht. Mithilfe einer Linse kann dieses Muster auf einen weiter entfernteren Schirm abgebildet werden. Das Muster auf dem weiter entfernteren Schirm ist dann jedoch von der Lage der Linse (deren Brennweite) abhängig. In der Abbildung \ref{fig:Fresnel} ist die Anordnung dieser Beobachtungsart zu sehen \cite{Angaben2011}.\\[-0.5cm]
\begin{figure}[h]
\begin{center}
\includegraphics[scale=0.8]{Bilder/Fresnel.png} 
\end{center}
\vspace*{-0.5cm}
\caption[Anordnung der Fresnel'schen Beobachtungsart]{\textbf{Anordnung der Fresnel'schen Beobachtungsart}: Die Wellen der Quellen (1, 2) treffen auf ein Objekt (z.B Spalt) im Nahfeld (a). Die Kugelwellen interferieren dann. Über eine Linse wird das so entstehende Interferenzmuster an einen Schirm (b) \glqq projiziert\grqq\; \cite{Angaben2011}.}
\label{fig:Fresnel}
\end{figure}
\newpage

\subsection{Frauenhofer'sche Beobachtungsart}
\label{subsec:frauenhofBeobachtungsart}
Ein Schirm wird in die Brennebene einer Linse gebracht, welches das Interferenzmuster darauf abbildet. Das somit resultierende Interferenzmuster ist hier nicht von der Lage der Linse zur Quelle abhängig. In der Abbildung \ref{fig:Frauenhofer} zu erkennen.\\
\begin{figure}[h]
\begin{center}
\includegraphics[scale=0.8]{Bilder/Frauenhofer.png}
\end{center}
\caption[Anordnung der Frauenhofer'schen Beobachtungsart]{\textbf{Anordnung der Frauenhofer'schen Beobachtungsart}: Das Interferenzmuster der Interferierenden Wellen wird über eine Linse im Fernfeld auf ein Schirm gegeben. Der Schirm steht in der Brennebene mit dem Abstand der Brennweite f der Linse. Das Interferenzmuster ist so unabhängig von der Linsenposition gegenüber dem Objekt \cite{Angaben2011}.}
\label{fig:Frauenhofer}
\end{figure}

Für die Beugung der Frauenhofer’sche Beobachtungsart gilt folgende Formel \ref{eq:1} \cite{Angaben2011}.\\
%FORMEL1 
\begin{equation}
a_{m}=f\cdot\tan\left( \arcsin\left[ \frac{m\cdot\lambda}{d}\right] \right) 
\label{eq:1}
\end{equation}
$a_{m}$ entspricht der Distanz des Maximums zur Mitte des Interferenzmusters. f der Brennweite. m ist repräsentativ für die Ordnungszahl. $\lambda$ steht für die Wellenlänge des He-Ne-Laserlichts (rot).\\

\subsection{Beugung am Spalt und Antispalt}
\label{subsec:beugspalt}
Beugung am Einzelspalt erzeugt im Fernfeld ein Interferenzmuster, bei welchem sich helle (Maximas) und dunkle (Minimas) Bereiche abwechseln. Die hellen Bereiche sind in der Mitte am hellsten und nach aussen gehend nimmt die Intensität ab. \cite{Angaben2011}.\\

\subsection{Beugung am Loch}
\label{subsec:beugloch}
Bei einer Lochblende wird als Interferenzmuster konzentrisch angelgte Ringe, Maxima und Minima abwechselnd auf dem Schirm beobachtet.\cite{Angaben2011}.\\[0.5cm]
Für die Auswertung bei der Beugung am Loch muss die Ordnungszahl der obigen Formel \ref{eq:1} angepasst werden, da es sich hier um Kreise handelt. Die neue Konstante, welche sich anstelle der Ordnungszahl einfügt, lässt sich gemäss folgender Formel \ref{eq:2} berechnen:
%FORMEL2
\begin{equation}
m_{i}=\frac{J_{1,i}}{\pi}
\label{eq:2}
\end{equation}

\subsubsection{Kreiskoeffizienten für Beugung am Loch}
Die ersten 9 Konstanten (m$_{i}$) werden für diese Arbeit nach der Formel \ref{eq:2} berechnet und unten dargestellt:\\[0.5cm]

\begin{tabular}[h]{ccc}
\centering
Koeffizientennummer & J$_{1,i}$ & m$_{i}$ \\ 
\hline 
1 & 3.832 & 1.220 \\  
2 & 7.016 & 2.233 \\  
3 & 10.173 & 3.238 \\ 
4 & 13.324 & 4.241 \\ 
5 & 16.471 & 5.243 \\ 
6 & 19.616 & 6.244 \\  
7 & 22.760 & 7.245 \\ 
8 & 25.904 & 8.245 \\ 
9 & 29.047 & 9.246 \\ 
\label{table:Koeff}
\end{tabular} 

\subsection{Beugung am Strichgitter}
\label{subsec:beugstrichgitter}
Bei der Beugung am Strichgitter ergibt sich ein Interferenzmuster, bei dem Lichtpunkte entlang der Horizontalen auftreten. Diese Lichtpunkte entstehen durch konstruktive Interferenz an diesen Stellen \cite{Angaben2011}.\\

\chapter{Durchführung}
% =================================================================
\thispagestyle{fancy}
% =================================================================
In diesem Kapitel werden der Messversuch und die verwendeten Mittel beschrieben.
\section{Versuchsanordnung}
Die Versuchsanordnung ist auf einer Zeiss-Schiene aufgebaut, auf der die Beugungsobjekte sowie die Linse für die Frauenhofer’sche Beobachtungsart fixiert werden können. Für jedes Objekt wird erst die direkte Beobachtungsart (ohne Linse), danach die Frauenhofer’sche Beobachtungsart (mit Linse) angewendet. Es werden verschiedene Linsen und Distanzen angewendet. Die zwei Beobachtungsarten sollen verglichen werden.\\[0.5cm]Als Lichtquelle steht ein He-Ne-Laser ($\lambda$ = 632.8 nm) zur Verfügung. Der Laser wird aufgeweitet verwendet. Die Distanzen auf der Zeiss-Schiene werden mit einem Doppelmeter gemessen, die Maxima/Minima mit Hilfe der Messeinrichtung an der Mattscheibe.\\
\begin{figure}[h]
\begin{center}
\includegraphics[width=\textwidth]{Bilder/Versuchsaufbau.png}
\end{center}
\caption[Versuchsaufbau]{Diese Abbildung zeigt den allgemeinen Versuchsaufbau. Links steht der Laser auf der Zeiss-Schiene und Rechts die Mattscheibe mit Messeinrichtung. Dazwischen sind die Objekte und Linsen zu platzieren\cite{TechnikFHNW2014}.}
\label{fig:Versuchsaufbau}
\end{figure}

\noindent
Geräte:
\begin{description}
\item[-]
He-Ne-Laser (rot, 632.8 $\cdot10^{-9}$m)
\item[-]
Doppelmeter (Annahme Abweichung: $s_{f}$ = 1$\cdot10^{-3}$m)
\item[-]
Messeinrichtung (Annahme Abweichung: $s_{a{m}}$= 1$\cdot10^{-3}$m)
\end{description}
\section{Messvorgang / Messmethoden}
\subsection{Beugung am Spalt und Antispalt}
Bei diesem Versuch werden 2 verschiedene Spalte und 2 verschiedene Antispalte je nacheinander auf der Zeiss-Schiene montiert und der Abstand der Beugungsmaxima/minima zum Mittelpunkt bestimmt. Mit diesen Werten wird dann die Spaltbreite/Antispaltdicke des Objekts mittels Regression an eine Funktionslinie ermittelt. Der Versuch wird mit beiden Beobachtungsarten durchgeführt und die Ergebnisse später verglichen.
\subsection{Beugung am Loch}
Bei diesem Versuch werden 2 verschiedene Löcher nacheinander auf der Zeiss-Schiene montiert und der Abstand der Beugungsminima zum Mittelpunkt bestimmt. Mit diesen Werten wird dann der Durchmesser der Löcher mittels Regression an eine Funktionslinie ermittelt. Dieser Versuch wird mit beiden Beobachtungsarten durchgeführt und die Ergebnisse später verglichen.
\subsection{Beugung am Strichgitter}
Bei diesem Versuch werden 2 verschiedene Strichgitter nacheinander auf der Zeiss-Schiene montiert und der Abstand der Beugungsmaxima zum Mittelpunkt bestimmt. Mit diesen Werten wird dann der Abstand der Gitterlinien zueinander mittels Regression an eine Funktionslinie ermittelt. Dieser Versuch wird mit beiden Beobachtungsarten durchgeführt und die Ergebnisse später verglichen.
\section{Versuchsobjekte}
\begin{tabular}{cccc}
Nr. & Objekt & Abmessung & Abweichung $s_{d}$ \\ 
\hline 
1 & Spalt 1& $200\cdot10^{-6}m$ & $\pm4\cdot10^{-6}m$\\ 
2 & Spalt 2& $150\cdot10^{-6}m$ & $\pm4\cdot10^{-6}m$ \\ 
3 & Antispalt 1& $530\cdot10^{-6}m$ & $\pm5\cdot10^{-6}m$ \\ 
4 & Antispalt 2& $430\cdot10^{-6}m$ & $\pm5\cdot10^{-6}m$ \\ 
5 & Loch 1& $150\cdot10^{-6}m$ & $\pm6\cdot10^{-6}m$ \\ 
6 & Loch 2& $100\cdot10^{-6}m$ & $\pm4\cdot10^{-6}m$ \\ 
7 & Strichgitter 1& $100\frac{lines}{mm}$ & keine Daten, Annahme: 1$\cdot 10^{-6}$m\\ 
8 & Strichgitter 2& $80\frac{lines}{mm}$ & keine Daten, Annahme: 1$\cdot 10^{-6}$m \\ 
\end{tabular} 
\section{Messungen}
Die Messresultate sind auf der Excel-Tabelle im Anhang zu finden.


\chapter{Auswertung}
\label{chap:auswertung}
% =================================================================
\thispagestyle{fancy}
% =================================================================
\section{Beugung am Spalt und Antispalt}
Hier wurden die Maximas, resp. die Minimas beim Interferenzmuster auf dem Schirm gemessen. Mithilfe der Gleichung \ref{eq:1} wurde dann die Spalt-, resp. die Antispaltbreite mit QTI-Plot gefittet und graphisch dargestellt. Die Wellenlänge l des He-Ne-Lasers wurde jeweils direkt in der Formel mit 6.328e-07 m, sowie die Brennweite f resp. der Abstand des Spaltes zum Schirm L eingetragen.\\
\vspace{-0.5cm}
%******************************************
\begin{figure}[h]
\centering
\includegraphics[width=\textwidth]{Bilder/spalt1_ohneLinse.png}
\vspace*{-3.5cm}
\caption[Spalt 1: ohne Linse]{Hier wird der Abstand vom Mittelpunkt zum Minima des Interferenzmusters bei bestimmter Ordnungszahl, direkter Beobachtung und ohne Linse dargestellt. Dabei wird, wie in der Legende zu sehen ist, die Spaltenbreite mit ihrem Fehler gefittet. Es wurde zum Fit die Distanz vom Objekt zum Schirm von 182cm verwendet.}
\label{fig:spalt1_ohneLinse}
\end{figure}
\newpage
%============================================
\begin{figure}[h]
\centering
\includegraphics[width=\textwidth]{Bilder/spalt1_mitLinse.png} 
\vspace*{-3.5cm}
\caption[Spalt 1: mit Linse]{Hier wird der Abstand vom Mittelpunkt zum Minima des Interferenzmusters bei bestimmter Ordnungszahl, Frauenhofer'scher Beobachtungsart und mit Linse dargestellt. Dabei wird, wie in der Legende zu sehen ist, die Spaltenbreite gefittet. Es wurde zum Fit die Distanz von der Linse zum Schirm von 100cm verwendet.}
\label{fig:spalt1_mitLinse}
\end{figure}
\newpage

\begin{figure}[h]
\centering
\includegraphics[width=\textwidth]{Bilder/spalt2_ohneLinse.png}
\vspace*{-1cm}
\caption[Spalt 2: ohne Linse]{Hier wird der Abstand vom Mittelpunkt zum Minima des Interferenzmusters bei bestimmter Ordnungszahl, direkter Beobachtung und ohne Linse dargestellt. Dabei wird, wie in der Legende zu sehen ist, die Spaltenbreite mit ihrem Fehler gefittet. Es wurde zum Fit die Distanz vom Objekt zum Schirm von 127.5cm verwendet.}
\label{fig:spalt2_ohneLinse}
\end{figure}
\newpage

\begin{figure}[h]
\centering
\includegraphics[width=\textwidth]{Bilder/spalt2_mitLinse.png} 
\vspace*{-1cm}
\caption[Spalt 2: mit Linse]{Hier wird der Abstand vom Mittelpunkt zum Minima des Interferenzmusters bei bestimmter Ordnungszahl, Frauenhofer'scher Beobachtungsart und mit Linse dargestellt. Dabei wird, wie in der Legende zu sehen ist, die Spaltenbreite gefittet. Es wurde zum Fit die Distanz von der Linse zum Schirm von 100cm verwendet.}
\label{fig:spalt2_mitLinse}
\end{figure}
\newpage

\begin{figure}[h]
\centering
\includegraphics[width=\textwidth]{Bilder/antispalt1_ohneLinse.png}
\vspace*{-3.5cm}
\caption[Antispalt 1: ohne Linse]{Hier wird der Abstand vom Mittelpunkt zum Minima des Interferenzmusters bei bestimmter Ordnungszahl, direkter Beobachtung und ohne Linse dargestellt. Dabei wird, wie in der Legende zu sehen ist, die Spaltenbreite mit ihrem Fehler gefittet. Es wurde zum Fit die Distanz vom Objekt zum Schirm von 186cm verwendet.}
\label{fig:antispalt1_ohneLinse}
\end{figure}
\newpage

\begin{figure}[h]
\centering
\includegraphics[width=\textwidth]{Bilder/antispalt1_mitLinse.png} 
\vspace*{-2cm}
\caption[Antispalt 1: mit Linse]{Hier wird der Abstand vom Mittelpunkt zum Minima des Interferenzmusters bei bestimmter Ordnungszahl, Frauenhofer'scher Beobachtungsart und mit Linse dargestellt. Dabei wird, wie in der Legende zu sehen ist, die Spaltenbreite gefittet. Es wurde zum Fit die Distanz von der Linse zum Schirm von 100cm verwendet.}
\label{fig:antispalt1_mitLinse}
\end{figure}
\newpage

\begin{figure}[h]
\centering
\includegraphics[width=\textwidth]{Bilder/antispalt2_ohneLinse.png}
\vspace*{-1cm}
\caption[Antispalt 2: ohne Linse]{Hier wird der Abstand vom Mittelpunkt zum Minima des Interferenzmusters bei bestimmter Ordnungszahl, direkter Beobachtung und ohne Linse dargestellt. Dabei wird, wie in der Legende zu sehen ist, die Spaltenbreite mit ihrem Fehler gefittet. Es wurde zum Fit die Distanz vom Objekt zum Schirm von 160.5cm verwendet.}
\label{fig:antispalt2_ohneLinse}
\end{figure}
\newpage

\begin{figure}[h]
\centering
\includegraphics[width=\textwidth]{Bilder/antispalt2_mitLinse.png} 
\vspace*{-1cm}
\caption[Antispalt 2: mit Linse]{Hier wird der Abstand vom Mittelpunkt zum Minima des Interferenzmusters bei bestimmter Ordnungszahl, Frauenhofer'scher Beobachtungsart und mit Linse dargestellt. Dabei wird, wie in der Legende zu sehen ist, die Spaltenbreite gefittet. Es wurde zum Fit die Distanz von der Linse zum Schirm von 100cm verwendet.}
\label{fig:antispalt2_mitLinse}
\end{figure}
\newpage

\section{Beugung am Loch}
Hier werden die horizontal zum Mittelpunkt liegenden Minimas der Ringe des Interferenzmusters bestimmt. Nach der Gleichung \ref{eq:1} wird dann der Durchmesser des Loches gefittet. Nach Empfehlung des Dozenten wurde das Antiloch weggelassen.

\begin{figure}[h]
\centering
\includegraphics[width=\textwidth]{Bilder/loch1_ohneLinse.png} 
\caption[Loch 1: ohne Linse]{Hier repräsentiert jeder Messpunkt den Abstand vom Nullpunkt zum Minima der Ringe des Interferenzmusters nach dem Durchschreiten des He-Ne-Lasers durch das Loch mit direkter Beobachtungsart und ohne Linse. Dabei wurden für die Ordnung die bereits eruierten Koeffizienten verwendet. Als Abstand zum Schirm wurde die Distanz vom Objekt (Loch) zum Schirm direkt in die Gleichung \ref{eq:1} von 132.8cm eingefügt.}
\label{fig:loch1_ohneLinse}
\end{figure}
\newpage
\begin{figure}[h]
\centering
\includegraphics[width=\textwidth]{Bilder/loch1_mitLinse.png}  
\caption[Loch 1: mit Linse]{Hier repräsentiert jeder Messpunkt den Abstand vom Nullpunkt zum Minima der Ringe des Interferenzmusters nach dem Durchschreiten des He-Ne-Lasers durch das Loch mit Frauenhofer'scher Beobachtungsart und mit Linse. Dabei wurden für die Ordnung die bereits eruierten Koeffizienten verwendet. Als Abstand zum Schirm wurde die Distanz von der Linse zum Schirm direkt in die Gleichung \ref{eq:1} von 100cm eingefügt.}
\label{fig:loch1_mitLinse}
\end{figure}
\newpage

\begin{figure}[h]
\centering
\includegraphics[width=\textwidth]{Bilder/loch2_ohneLinse.png} 
\caption[Loch 2: ohne Linse]{Hier repräsentiert jeder Messpunkt den Abstand vom Nullpunkt zum Minima der Ringe des Interferenzmusters nach dem Durchschreiten des He-Ne-Lasers durch das Loch mit direkter Beobachtungsart und ohne Linse. Dabei wurden für die Ordnung die bereits eruierten Koeffizienten verwendet. Als Abstand zum Schirm wurde die Distanz vom Objekt (Loch) zum Schirm direkt in die Gleichung \ref{eq:1} von 146.5cm eingefügt.}
\label{fig:loch2_ohneLinse}
\end{figure}
\newpage
\begin{figure}[h]
\centering
\includegraphics[width=\textwidth]{Bilder/loch2_mitLinse.png}  
\caption[Loch 2: mit Linse]{Hier repräsentiert jeder Messpunkt den Abstand vom Nullpunkt zum Minima der Ringe des Interferenzmusters nach dem Durchschreiten des He-Ne-Lasers durch das Loch mit Frauenhofer'scher Beobachtungsart und mit Linse. Dabei wurden für die Ordnung die bereits eruierten Koeffizienten verwendet. Als Abstand zum Schirm wurde die Distanz von der Linse zum Schirm direkt in die Gleichung \ref{eq:1} von 100cm eingefügt.}
\label{fig:loch2_mitLinse}
\end{figure}

\section{Beugung am Strichgitter}
Bei den Auswertungen/Messungen der Beugung des Lichts am Strichgitter traten Komplikationen auf, womit diese Ergebnisse zu keinem plausiblen Resultat führten. Die Plots sind im Anhang im im Kapitel \ref{sec:BeugungAmStrichgitter} \textit{\nameref{sec:BeugungAmStrichgitter}} zu finden
\newpage

\chapter{Fehlerrechnung}
% =================================================================
\thispagestyle{fancy}
% =================================================================
\section{Statistischer Fehler}
Die Aufsummierung der quadrierten partiellen Ableitungen (nach den fehlerbehafteten Grössen) als Diskriminante unter der Wurzel ergibt den statistischen Fehler.\\
\begin{equation}
s=\sqrt{\left( \frac{\partial d}{\partial f} \cdot s_{f}\right)^{2} +\left( \frac{\partial d}{\partial a_{m}} \cdot s_{a_{m}}\right)^{2}}
\label{eq:p3}
\end{equation}

\section{Systematischer Fehler}
Das einige Element in diesem Versuch, welche eine systematische Fehlerquelle wäre, ist der He-Ne-Laser. Da dieser nach \textit{Leif Physik} \cite{PhysikkeineAngaben} mit den höheren und tieferen Energieniveaus der Helium- und Neonatomen arbeitet, ist dessen Unsicherheit auf die Wellenlänge des roten Lichts vernachlässigbar klein. 
\section{Absoluter Fehler}
Statistischer und systematischer Fehler werden quadriert und als Diskriminante unter der Wurzel aufsummiert. Daraus ergibt sich der absolute Fehler jeder Messung. Da aber der systematische Fehler vernachlässigt wird, wäre der absolute Fehler equivalent zum statistischen Fehler.

\chapter{Resultate und Diskussion}
% =================================================================
\thispagestyle{fancy}
% =================================================================

\chapter{Begriffsexplikation}
% =================================================================
\thispagestyle{fancy}
\label{chap:begriffsexplikation}
% =================================================================
\section*{Resonanz}
\glqq Schwingende Körper (Schwinger) können durch Energiezufuhr von außen zu erzwungenen Schwingungen angeregt werden. Ist die Erregerfrequenz gleich der Eigenfrequenz des Schwingers, so erreicht die Amplitude der Schwingung ein Maximum.\grqq\: \cite{resonanz}\\
Ist aber die Anregung grösser als die vorhandene Dämpfung, kann es das System wahrlich zerbersten. Dieses Phänomen wird \textbf{Resonanzkatastrophe} genannt. \\
Dafür ein kleines Videobeispiel in diesem Link: \url{https://www.youtube.com/watch?v=lXyG68_caV4}

\section*{Eigenfrequenz}
\glqq Die Eigenfrequenz ist die Frequenz, mit der technische Schwingsysteme mit einer bewegten Masse und einem Freiheitsgrad der Bewegung nach einer einmaligen Anregung schwingen. Dabei schwingt das System immer in charakteristischen Eigenfrequenzen erster und höherer Ordnung.\grqq\: \cite{eigenfrequenz}\\
Als Beispiel dafür kann eine Kinderschaukel betrachtet werden. Die Eigenfrequenz bleibt immer gleich, solange die gleichen Bedingungen gelten (z. B. Gewicht des Kindes und/oder der Schaukel). Egal wie hoch das Kind schaukelt, die Frequenz mit der das Kind durch die Ruhelage hindurch schaukelt, bleibt die gleiche.

\chapter{Plagiatserklärung}
% =================================================================
\thispagestyle{fancy}
% =================================================================
Ich, Andres Minder, der Versuchsleiter in diesem Versuch versichere, dass dieses Laborjournal selbstständig erarbeitet wurde. Alle Quellen und Hilfsmittel aus anderen Werken, die dem Wortlaut oder dem Sinne nach entnommen wurden und zu dieser Arbeit beigetragen haben, sind jeweils kenntlich referenziert.\\
\vfill
\begin{center}
\begin{tabular}{p{5cm}p{1cm}l}
\Large\textbf{Ort, Datum:} & & \Large\textbf{Unterschrift des Versuchsleiters:} \\
\vspace{1cm} & \vspace{1cm} & \vspace{1cm} \\
\centering\huge\today & & \\
\hrulefill & & \hrulefill 
\end{tabular}
\end{center}

%Literaturverzeichnis
\addcontentsline{toc}{chapter}{Literaturverzeichnis}
\bibliography{Literaturverzeichnis/lit_w6}

\chapter*{Anhang}
% =================================================================
\thispagestyle{fancy} \addcontentsline{toc}{chapter}{Anhang}
% =================================================================
\section*{Messresultate \glqq Bestimmen der Eigenfrequenz\grqq}
\begin{figure}[h]
\centering
\includegraphics[scale=0.8]{Bilder/messungen_eigenfrequenz.png} 
\caption{Messresultate mit den gewichteten und ungewichteten Mittelwerten und Fehlern}
\label{fig:messresultate_bestimmen_eigenfrequenz}
\end{figure}
\section*{Messresultate \glqq Amplitudenverlauf\grqq}
\begin{figure}[h]
\centering
\includegraphics[scale=0.73]{Bilder/messungen_amplitudenverlauf.png} 
\caption{Amplitudenverlauf mit dem Kalibrierungsfaktor um die Zeitdaten anzupassen wegen des schlechten Clocks des Laser Distanzmessgeräts}
\label{fig:messresultate_amplitudenverlauf}
\end{figure}
\newpage
\section*{Messresultate \glqq Amplituden- und Phasenresonanz\grqq}
\begin{figure}[h]
\centering
\includegraphics[scale=1.3]{Bilder/messungen_phasenresonanz.png} 
\caption{Messresultate für die Bestimmung der Amplituden- und Phasenresonanz}
\label{fig:messresultate_phasenresonanz}
\end{figure}
\section*{Amplitudenverläufe}
\begin{figure}[h]
\centering
\includegraphics[scale=1]{Bilder/amplitudenverlauf_005_4.png} 
\caption{Amplitudenverlauf; Startauslenkung: $0.05m$, Anzahl Magnete: 4}
\label{fig:amplitudenverlauf_005_4}
\end{figure}
\newpage
\begin{figure}[h]
\centering
\includegraphics[scale=1.1]{Bilder/amplitudenverlauf_01_4.png} 
\caption{Amplitudenverlauf; Startauslenkung: $0.1m$, Anzahl Magnete: 4}
\label{fig:amplitudenverlauf_01_4}
\end{figure}
\newpage
\begin{figure}[h]
\centering
\includegraphics[scale=1.2]{Bilder/amplitudenverlauf_005_6.png} 
\caption{Amplitudenverlauf; Startauslenkung: $0.05m$, Anzahl Magnete: 6}
\label{fig:amplitudenverlauf_005_6}
\end{figure}
\newpage

\end{document}