\chapter{Durchführung}
% =================================================================
\thispagestyle{fancy}
% =================================================================
\section{Versuchsanordnung}
Auf einer Zeiss-Schiene wird das Ganze aufgebaut, auf der die Beugungsobjekte sowie die Linse für die Frauenhofer’sche Beobachtungsart fixiert werden können. Bei jedem Objekt wird erst die direkte Beobachtungsart (jene ohne Linse), anschließend die Frauenhofer’sche Beobachtungsart (jene mit Linse) angewendet. Die zwei Beobachtungsarten werden im Kapitel \nameref{chap:resultateDiskussion} dann mit der Wertvorgabe verglichen.
\\[0.5cm]
Als Lichtquelle steht ein He-Ne-Laser ($\lambda$ = 632.8 nm) zur Verfügung. Mit einem Doppelmeter werden die Distanzen auf der Zeiss-Schiene gemessen. Die Maximas, resp. die Minimas mit Hilfe der Messeinrichtung hinten an der Mattscheibe.\\
\begin{figure}[h]
\begin{center}
\includegraphics[width=\textwidth]{Bilder/Versuchsaufbau.png}
\end{center}
\caption[Versuchsaufbau]{\textbf{Versuchsaufbau}: Zwischen dem He-Ne-Laser und der Mattscheibe mit Messeinrichtung werden die Objekte im Objekthalter befestigt, sowie die Linse mit dem Abstand der Brennweite zur Mattscheibe eingerichtet \cite{TechnikFHNW2014}.}
\label{fig:Versuchsaufbau}
\end{figure}

\noindent
Geräte:
\begin{description}
\item[-]
He-Ne-Laser (rot, 632.8 $\cdot10^{-9}$m)
\item[-]
Doppelmeter (Annahme Abweichung: $s_{f}$ = 1$\cdot10^{-3}$m)
\item[-]
Messeinrichtung (Annahme Abweichung: $s_{a{m}}$= 1$\cdot10^{-3}$m)
\end{description}
\newpage

\section{Messvorgang / Messmethoden}
\subsection{Beugung am Spalt und Antispalt}
Hier werden zwei verschiedene Spalte und zwei verschiedene Antispalte nacheinander auf der Zeiss-Schiene montiert und der Abstand der Beugungsmaximas/minimas zum Mittelpunkt hinten auf der Mattscheibe abgelesen. Anschliessend wird die Spaltendicke d mittels Regression mit QTI-Plot bestimmt. Es werden beide Beobachtungsarten für die Messreihen verwendet.
\subsection{Beugung am Loch}
Es werden zwei verschiedene Löcher nacheinander auf der Zeiss-Schiene montiert und der Abstand der Beugungsminima zum Mittelpunkt bestimmt. Dabei wird dann der Lochdurchmesser des Loches bestimmt. Es werden beide Beobachtungsarten für die Messreihen verwendet.
\subsection{Beugung am Strichgitter}
Es werden zwei verschiedene Strichgitter nacheinander auf der Zeiss-Schiene montiert und der Abstand der Beugungsmaxima zum Mittelpunkt bestimmt. Der Gitterlinienabstand wird dann anschließend bestimmt. Es werden beide Beobachtungsarten für die Messreihen verwendet.
\section{Versuchsobjekte}
\begin{tabular}{|l|l|l|}
\toprule
Objekt & Abmessung & Abweichung $s_{d}$ \\ 
\toprule
Spalt 1& $200\cdot10^{-6}m$ & $\pm4\cdot10^{-6}m$\\ 
Spalt 2& $150\cdot10^{-6}m$ & $\pm4\cdot10^{-6}m$ \\ 
Antispalt 1& $530\cdot10^{-6}m$ & $\pm5\cdot10^{-6}m$ \\ 
Antispalt 2& $430\cdot10^{-6}m$ & $\pm5\cdot10^{-6}m$ \\ 
Loch 1& $150\cdot10^{-6}m$ & $\pm6\cdot10^{-6}m$ \\ 
Loch 2& $100\cdot10^{-6}m$ & $\pm4\cdot10^{-6}m$ \\ 
Strichgitter 1& $100\frac{lines}{mm}$ & keine Angaben \\ 
Strichgitter 2& $80\frac{lines}{mm}$ & keine Angaben \\ 
\bottomrule
\end{tabular} 