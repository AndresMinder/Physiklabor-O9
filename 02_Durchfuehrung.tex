\chapter{Durchführung}
% =================================================================
\thispagestyle{fancy}
% =================================================================
In diesem Kapitel werden der Messversuch und die verwendeten Mittel beschrieben.
\section{Versuchsanordnung}
Die Versuchsanordnung ist auf einer Zeiss-Schiene aufgebaut, auf der die Beugungsobjekte sowie die Linse für die Frauenhofer’sche Beobachtungsart fixiert werden können. Für jedes Objekt wird erst die direkte Beobachtungsart (ohne Linse), danach die Frauenhofer’sche Beobachtungsart (mit Linse) angewendet. Es werden verschiedene Linsen und Distanzen angewendet. Die zwei Beobachtungsarten sollen verglichen werden.\\[0.5cm]Als Lichtquelle steht ein He-Ne-Laser ($\lambda$ = 632.8 nm) zur Verfügung. Der Laser wird aufgeweitet verwendet. Die Distanzen auf der Zeiss-Schiene werden mit einem Doppelmeter gemessen, die Maxima/Minima mit Hilfe der Messeinrichtung an der Mattscheibe.\\
\begin{figure}[h]
\begin{center}
\includegraphics[width=\textwidth]{Bilder/Versuchsaufbau.png}
\end{center}
\caption[Versuchsaufbau]{Diese Abbildung zeigt den allgemeinen Versuchsaufbau. Links steht der Laser auf der Zeiss-Schiene und Rechts die Mattscheibe mit Messeinrichtung. Dazwischen sind die Objekte und Linsen zu platzieren\cite{TechnikFHNW2014}.}
\label{fig:Versuchsaufbau}
\end{figure}

\noindent
Geräte:
\begin{description}
\item[-]
He-Ne-Laser (rot, 632.8 $\cdot10^{-9}$m)
\item[-]
Doppelmeter (Annahme Abweichung: $s_{f}$ = 1$\cdot10^{-3}$m)
\item[-]
Messeinrichtung (Annahme Abweichung: $s_{a{m}}$= 1$\cdot10^{-3}$m)
\end{description}
\section{Messvorgang / Messmethoden}
\subsection{Beugung am Spalt und Antispalt}
Bei diesem Versuch werden 2 verschiedene Spalte und 2 verschiedene Antispalte je nacheinander auf der Zeiss-Schiene montiert und der Abstand der Beugungsmaxima/minima zum Mittelpunkt bestimmt. Mit diesen Werten wird dann die Spaltbreite/Antispaltdicke des Objekts mittels Regression an eine Funktionslinie ermittelt. Der Versuch wird mit beiden Beobachtungsarten durchgeführt und die Ergebnisse später verglichen.
\subsection{Beugung am Loch}
Bei diesem Versuch werden 2 verschiedene Löcher nacheinander auf der Zeiss-Schiene montiert und der Abstand der Beugungsminima zum Mittelpunkt bestimmt. Mit diesen Werten wird dann der Durchmesser der Löcher mittels Regression an eine Funktionslinie ermittelt. Dieser Versuch wird mit beiden Beobachtungsarten durchgeführt und die Ergebnisse später verglichen.
\subsection{Beugung am Strichgitter}
Bei diesem Versuch werden 2 verschiedene Strichgitter nacheinander auf der Zeiss-Schiene montiert und der Abstand der Beugungsmaxima zum Mittelpunkt bestimmt. Mit diesen Werten wird dann der Abstand der Gitterlinien zueinander mittels Regression an eine Funktionslinie ermittelt. Dieser Versuch wird mit beiden Beobachtungsarten durchgeführt und die Ergebnisse später verglichen.
\section{Versuchsobjekte}
\begin{tabular}{cccc}
Nr. & Objekt & Abmessung & Abweichung $s_{d}$ \\ 
\hline 
1 & Spalt 1& $200\cdot10^{-6}m$ & $\pm4\cdot10^{-6}m$\\ 
2 & Spalt 2& $150\cdot10^{-6}m$ & $\pm4\cdot10^{-6}m$ \\ 
3 & Antispalt 1& $530\cdot10^{-6}m$ & $\pm5\cdot10^{-6}m$ \\ 
4 & Antispalt 2& $430\cdot10^{-6}m$ & $\pm5\cdot10^{-6}m$ \\ 
5 & Loch 1& $150\cdot10^{-6}m$ & $\pm6\cdot10^{-6}m$ \\ 
6 & Loch 2& $100\cdot10^{-6}m$ & $\pm4\cdot10^{-6}m$ \\ 
7 & Strichgitter 1& $100\frac{lines}{mm}$ & keine Daten, Annahme: 1$\cdot 10^{-6}$m\\ 
8 & Strichgitter 2& $80\frac{lines}{mm}$ & keine Daten, Annahme: 1$\cdot 10^{-6}$m \\ 
\end{tabular} 
\section{Messungen}
Die Messresultate sind auf der Excel-Tabelle im Anhang zu finden.
