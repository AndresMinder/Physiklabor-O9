\chapter{Auswertung}
% =================================================================
\thispagestyle{fancy}
% =================================================================
\section{Beugung am Spalt und Antispalt}
Hier wurden die Maximas, resp. die Minimas beim Interferenzmuster auf dem Schirm gemessen. Mithilfe der Gleichung \ref{eq:1} wurde dann die Spalt-, resp. die Antispaltbreite mit QTI-Plot gefittet und graphisch dargestellt. Die Wellenlänge l des He-Ne-Lasers wurde jeweils direkt in der Formel mit 6.328e-07 m, sowie die Brennweite f resp. der Abstand des Spaltes zum Schirm L eingetragen.\\
\vspace{-0.5cm}
%******************************************
\begin{figure}[h]
\centering
\includegraphics[width=\textwidth]{Bilder/spalt1_ohneLinse.png}
\vspace*{-3.5cm}
\caption[Spalt 1: ohne Linse]{Hier wird der Abstand vom Mittelpunkt zum Minima des Interferenzmusters bei bestimmter Ordnungszahl, direkter Beobachtung und ohne Linse dargestellt. Dabei wird, wie in der Legende zu sehen ist, die Spaltenbreite mit ihrem Fehler gefittet. Es wurde zum Fit die Distanz vom Objekt zum Schirm von 182cm verwendet.}
\label{fig:spalt1_ohneLinse}
\end{figure}
\newpage
%============================================
\begin{figure}[h]
\centering
\includegraphics[width=\textwidth]{Bilder/spalt1_mitLinse.png} 
\vspace*{-3.5cm}
\caption[Spalt 1: mit Linse]{Hier wird der Abstand vom Mittelpunkt zum Minima des Interferenzmusters bei bestimmter Ordnungszahl, Frauenhofer'scher Beobachtungsart und mit Linse dargestellt. Dabei wird, wie in der Legende zu sehen ist, die Spaltenbreite gefittet. Es wurde zum Fit die Distanz von der Linse zum Schirm von 100cm verwendet.}
\label{fig:spalt1_mitLinse}
\end{figure}
\newpage

\begin{figure}[h]
\centering
\includegraphics[width=\textwidth]{Bilder/spalt2_ohneLinse.png}
\vspace*{-1cm}
\caption[Spalt 2: ohne Linse]{Hier wird der Abstand vom Mittelpunkt zum Minima des Interferenzmusters bei bestimmter Ordnungszahl, direkter Beobachtung und ohne Linse dargestellt. Dabei wird, wie in der Legende zu sehen ist, die Spaltenbreite mit ihrem Fehler gefittet. Es wurde zum Fit die Distanz vom Objekt zum Schirm von 127.5cm verwendet.}
\label{fig:spalt2_ohneLinse}
\end{figure}
\newpage

\begin{figure}[h]
\centering
\includegraphics[width=\textwidth]{Bilder/spalt2_mitLinse.png} 
\vspace*{-1cm}
\caption[Spalt 2: mit Linse]{Hier wird der Abstand vom Mittelpunkt zum Minima des Interferenzmusters bei bestimmter Ordnungszahl, Frauenhofer'scher Beobachtungsart und mit Linse dargestellt. Dabei wird, wie in der Legende zu sehen ist, die Spaltenbreite gefittet. Es wurde zum Fit die Distanz von der Linse zum Schirm von 100cm verwendet.}
\label{fig:spalt2_mitLinse}
\end{figure}
\newpage

\begin{figure}[h]
\centering
\includegraphics[width=\textwidth]{Bilder/antispalt1_ohneLinse.png}
\vspace*{-3.5cm}
\caption[Antispalt 1: ohne Linse]{Hier wird der Abstand vom Mittelpunkt zum Minima des Interferenzmusters bei bestimmter Ordnungszahl, direkter Beobachtung und ohne Linse dargestellt. Dabei wird, wie in der Legende zu sehen ist, die Spaltenbreite mit ihrem Fehler gefittet. Es wurde zum Fit die Distanz vom Objekt zum Schirm von 186cm verwendet.}
\label{fig:antispalt1_ohneLinse}
\end{figure}
\newpage

\begin{figure}[h]
\centering
\includegraphics[width=\textwidth]{Bilder/antispalt1_mitLinse.png} 
\vspace*{-2cm}
\caption[Antispalt 1: mit Linse]{Hier wird der Abstand vom Mittelpunkt zum Minima des Interferenzmusters bei bestimmter Ordnungszahl, Frauenhofer'scher Beobachtungsart und mit Linse dargestellt. Dabei wird, wie in der Legende zu sehen ist, die Spaltenbreite gefittet. Es wurde zum Fit die Distanz von der Linse zum Schirm von 100cm verwendet.}
\label{fig:antispalt1_mitLinse}
\end{figure}
\newpage

\begin{figure}[h]
\centering
\includegraphics[width=\textwidth]{Bilder/antispalt2_ohneLinse.png}
\vspace*{-1cm}
\caption[Antispalt 2: ohne Linse]{Hier wird der Abstand vom Mittelpunkt zum Minima des Interferenzmusters bei bestimmter Ordnungszahl, direkter Beobachtung und ohne Linse dargestellt. Dabei wird, wie in der Legende zu sehen ist, die Spaltenbreite mit ihrem Fehler gefittet. Es wurde zum Fit die Distanz vom Objekt zum Schirm von 160.5cm verwendet.}
\label{fig:antispalt2_ohneLinse}
\end{figure}
\newpage

\begin{figure}[h]
\centering
\includegraphics[width=\textwidth]{Bilder/antispalt2_mitLinse.png} 
\vspace*{-1cm}
\caption[Antispalt 2: mit Linse]{Hier wird der Abstand vom Mittelpunkt zum Minima des Interferenzmusters bei bestimmter Ordnungszahl, Frauenhofer'scher Beobachtungsart und mit Linse dargestellt. Dabei wird, wie in der Legende zu sehen ist, die Spaltenbreite gefittet. Es wurde zum Fit die Distanz von der Linse zum Schirm von 100cm verwendet.}
\label{fig:antispalt2_mitLinse}
\end{figure}
\newpage

\section{Beugung am Loch}