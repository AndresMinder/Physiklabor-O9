\chapter{Fehlerrechnung}
% =================================================================
\thispagestyle{fancy}
% =================================================================
\section{Statistischer Fehler}
Die Aufsummierung der quadrierten partiellen Ableitungen (nach den fehlerbehafteten Grössen) als Diskriminante unter der Wurzel ergibt den statistischen Fehler.\\
\begin{equation}
s=\sqrt{\left( \frac{\partial d}{\partial f} \cdot s_{f}\right)^{2} +\left( \frac{\partial d}{\partial a_{m}} \cdot s_{a_{m}}\right)^{2}}
\label{eq:p3}
\end{equation}

\section{Systematischer Fehler}
Das einige Element in diesem Versuch, welche eine systematische Fehlerquelle wäre, ist der He-Ne-Laser. Da dieser nach \textit{Leif Physik} \cite{PhysikkeineAngaben} mit den höheren und tieferen Energieniveaus der Helium- und Neonatomen arbeitet, ist dessen Unsicherheit auf die Wellenlänge des roten Lichts vernachlässigbar klein. 
\section{Absoluter Fehler}
Statistischer und systematischer Fehler werden quadriert und als Diskriminante unter der Wurzel aufsummiert. Daraus ergibt sich der absolute Fehler jeder Messung. Da aber der systematische Fehler vernachlässigt wird, wäre der absolute Fehler equivalent zum statistischen Fehler.