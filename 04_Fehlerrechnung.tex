\chapter{Fehlerrechnung}
% =================================================================
\thispagestyle{fancy}
% =================================================================
Da in den Auswertungen nie irgendwelche Konstanten (z. B. von systematischen Grössen) verwendet wurden, sind in dieser Fehlerrechnung nur die statistischen Werte miteinbezogen. Die Unsicherheiten der Messgeräte wurden vernachlässigt\footnote{teils wegen fehlenden Angaben}. Alle hier benutzten Formeln sind aus den Arbeitsgrundlagen für das Grundlagenlabor 4 \cite{TechnikFHNW2016}. \\
\\[0.25cm]
\noindent\rule{\textwidth}{0.4pt}
\\[0.4cm]
Als erstes soll der Fehler der Eigenfrequenzbestimmung berechnet werden:\\
Dafür wurde zuerst der ungewogene Fehler des Mittelwertes von $\omega$ für jede Messreihe berechnet. Deren Indizes sind nach der Startauslenkung angegeben.
\begin{table}[H]
\centering
\begin{tabular}{cc}
$s_{\bar{\omega}25}=\sqrt{\frac{\sum\nolimits_{i=1}^{10}[\omega_{i25}-4.866651691
]^{2}}{90}}=\pm\uline{5.594812E-3}
$ & $s_{\bar{\omega}50}=\sqrt{\frac{\sum\nolimits_{i=1}^{10}[\omega_{i50}-4.868005082
]^{2}}{90}}=\pm\uline{4.922105E-3}
$ \\ 
$s_{\bar{\omega}75}=\sqrt{\frac{\sum\nolimits_{i=1}^{10}[\omega_{i75}-4.856298548
]^{2}}{90}}=\pm\uline{1.94677E-3}
$ & $s_{\bar{\omega}100}=\sqrt{\frac{\sum\nolimits_{i=1}^{10}[\omega_{i100}-4.82146558
]^{2}}{90}}=\pm\uline{5.57849E-05}
$ \\ 
\end{tabular} 
\end{table}
Um dann einen möglichst genauen Wert zu erhalten, wurde der gewogene Mittelwert berechnet mit einem Fehler von:
\begin{table}[H]
\centering
\begin{tabular}{c}
{\Large $s_{\bar{\omega}}=\frac{1}{\sqrt{(0.005594812)^{-2}+(0.004922105)^{-2}+(0.00194677)^{-2}+(5.57849E-05)^{-2}}}$}$=\pm\uuline{1.11511E-04}\;rad/s$
\end{tabular} 
\end{table}
\noindent\rule{\textwidth}{0.4pt}
\\[0.4cm]
Bei der Messung des Amplitudenverlaufs wurde aus den fehlerbehafteten Grössen $\Gamma$ und $\omega$ die Kreisfrequenz $\omega_{0}$ berechnet. Nach dem \textit{Gauss'schen Fehlerfortpflanzungsgesetz} wird hier nun für die Fehlerbalken in der Abbildung \ref{fig:omega_0} die Berechnung gezeigt.\\
Dafür wurde zuerst die Gleichung \ref{equ:omega} nach $\omega_{0}$ umgeformt und nach den zwei fehlerbehafteten Grössen partiell differenziert. Die verwendeten Fehlergrössen wurden von der QTI-Plot Software berechnet und sind in den Abbildungen \ref{fig:amplitudenverlauf_01_6} \& \ref{fig:amplitudenverlauf_005_4} \& \ref{fig:amplitudenverlauf_01_4} \& \ref{fig:amplitudenverlauf_005_6} aus der Legende herauszulesen.
\begin{table}[H]
\centering
\begin{tabular}{c}
{\large $s_{\omega_{0}}=\sqrt{(\frac{d\omega_{0}}{d\Gamma}\bigg\vert_{\bar{\omega_{0}}}*s_{\bar{\Gamma}})^{2}+(\frac{d\omega_{0}}{d\omega}\bigg\vert_{\bar{\omega_{0}}}*s_{\bar{\omega}})^{2}}$}
\end{tabular} 
\end{table}
Nun müssen nur noch für die vier verschiedenen Runs die Funktions-, sowie die Fehlerwerte aus den Abbildungen herausgelesen und in diese Gleichung eingefügt werden.
\begin{itemize}
\centering
\item \textbf{Run 1}: $s_{\omega_{0}}=\pm\uuline{4.16884333E-04}\;rad/s$
\item \textbf{Run 2}: $s_{\omega_{0}}=\pm\uuline{1.88903774E-04}\;rad/s$
\item \textbf{Run 3}: $s_{\omega_{0}}=\pm\uuline{3.00180726E-04}\;rad/s$
\item \textbf{Run 4}: $s_{\omega_{0}}=\pm\uuline{1.32542824E-04}\;rad/s$
\end{itemize}
\newpage
\noindent\rule{\textwidth}{0.4pt}
\\[0.4cm]
Die Amplitudenresonanz enthält drei fehlerbehaftete Grössen. $\omega_{0}$, $\Gamma$ und $\hat{y}_{e}$. Mit QTI-Plot sind die Grössen gefittet worden und in der Abbildung \ref{fig:amplitudenresonanz} herauslesbar. Anhand der dort jeweils berechneten Grössen wird der Fehler der Aplitude wieder nach dem \textit{Gauss'schen Fehlerfortpflanzungsgesetz} berechnet:
\begin{table}[H]
\centering
\begin{tabular}{c}
{\Large $s_{\hat{y}}=\sqrt{(\frac{d\hat{y}}{d\Gamma}\bigg\vert_{\bar{\hat{y}}}*s_{\bar{\Gamma}})^{2} + (\frac{d\hat{y}}{d\omega_{0}}\bigg\vert_{\bar{\hat{y}}}*s_{\bar{\omega_{0}}})^{2} + (\frac{d\hat{y}}{d\hat{y}_{e}}\bigg\vert_{\bar{\hat{y}}}*s_{\bar{\hat{y}_{e}}})^{2}}$}
\end{tabular} 
\end{table}
Wie die Gleichung \ref{equ:erzwungene} von $\Omega$ abhängig ist, ist jetzt auch dessen Fehler von $\Omega$ abhängig, da kein Mittelwert für die Amplitudenwerte berechnet werden konnte. Nach einsetzen der verwendeten Erregerkreisfrequenzen $\Omega$ ergibt sich:
\begin{table}[H]
\centering
\begin{tabular}{lcl}
$\Omega= 3.769911184$ & $\rightarrow$ & $s_{\hat{y}}=\pm\uuline{2.88297665E-06}\;m$\\
$\Omega= 4.71238898$  & $\rightarrow$ & $s_{\hat{y}}=\pm\uuline{5.67110104E-05}\;m$\\
$\Omega= 5.026548246$ & $\rightarrow$ & $s_{\hat{y}}=\pm\uuline{1.43365779E-05}\;m$\\
$\Omega= 5.654866776$ & $\rightarrow$ & $s_{\hat{y}}=\pm\uuline{2.92953597E-06}\;m$\\
\end{tabular} 
\end{table}
Dasselbe wurde mit den Messwerten für die Phasenresonanz gemacht. Hier sind zwei fehlerbehaftete Grössen vorhanden, nach denen die Gleichung \ref{equ:phasenresonanz} gefittet wurde. $\Gamma$ und $\omega_{0}$.
\begin{table}[H]
\centering
\begin{tabular}{c}
{\Large $s_{\delta}=\sqrt{(\frac{d\delta}{d\Gamma}\bigg\vert_{\bar{\delta}}*s_{\bar{\Gamma}})^{2} + (\frac{d\delta}{d\omega_{0}}\bigg\vert_{\bar{\delta}}*s_{\bar{\omega_{0}}})^{2}}$}
\end{tabular} 
\end{table}
Nach einsetzen der Werte aus der Abbildung \ref{fig:phasenresonanz} ergibt sich daraus:
\begin{table}[H]
\centering
\begin{tabular}{lcl}
$\Omega= 3.769911184$ & $\rightarrow$ & $s_{\delta}=\pm\uuline{1.56228858E-02}\;rad/s$\\
$\Omega= 4.71238898$  & $\rightarrow$ & $s_{\delta}=\pm\uuline{9.67339285E-02}\;rad/s$\\
$\Omega= 5.026548246$ & $\rightarrow$ & $s_{\delta}=\pm\uuline{5.74452374E-02}\;rad/s$\\
$\Omega= 5.654866776$ & $\rightarrow$ & $s_{\delta}=\pm\uuline{2.06653692E-02}\;rad/s$\\
\end{tabular} 
\end{table}
Diese Werte entsprechen den Unsicherheiten für die einzelnen Messpunkte!\\
\noindent\rule{\textwidth}{0.4pt}
\\[0.4cm]
Zusätzlich muss noch der absolute Fehler von $\omega_{0}$ von der Amplituden- und der Phasenresonanz eruiert werden:
\begin{table}[H]
\centering
\begin{tabular}{ccc}
$s_{\omega_{0}}=\sqrt{s_{\omega_{0A}}^{2}+s_{\omega_{0P}}^{2}}$ & $\rightarrow$ & $s_{\omega_{0}}=\pm\uuline{9.615603575E-03}\;rad/s$
\end{tabular} 
\end{table}
\noindent\rule{\textwidth}{0.4pt}
\\[0.4cm]
\begin{center}
\fbox{\colorbox{yellow}{\textbf{Die berechneten Fehler sind alle nur \uline{statistisch}!!}}}
\end{center}
Können aber als absolut betrachtet werden wegen der Vernachlässigung systematischer Unsicherheiten.
\newpage