\chapter{Begriffsexplikation}
% =================================================================
\thispagestyle{fancy}
\label{chap:begriffsexplikation}
% =================================================================
\section*{Resonanz}
\glqq Schwingende Körper (Schwinger) können durch Energiezufuhr von außen zu erzwungenen Schwingungen angeregt werden. Ist die Erregerfrequenz gleich der Eigenfrequenz des Schwingers, so erreicht die Amplitude der Schwingung ein Maximum.\grqq\: \cite{resonanz}\\
Ist aber die Anregung grösser als die vorhandene Dämpfung, kann es das System wahrlich zerbersten. Dieses Phänomen wird \textbf{Resonanzkatastrophe} genannt. \\
Dafür ein kleines Videobeispiel in diesem Link: \url{https://www.youtube.com/watch?v=lXyG68_caV4}

\section*{Eigenfrequenz}
\glqq Die Eigenfrequenz ist die Frequenz, mit der technische Schwingsysteme mit einer bewegten Masse und einem Freiheitsgrad der Bewegung nach einer einmaligen Anregung schwingen. Dabei schwingt das System immer in charakteristischen Eigenfrequenzen erster und höherer Ordnung.\grqq\: \cite{eigenfrequenz}\\
Als Beispiel dafür kann eine Kinderschaukel betrachtet werden. Die Eigenfrequenz bleibt immer gleich, solange die gleichen Bedingungen gelten (z. B. Gewicht des Kindes und/oder der Schaukel). Egal wie hoch das Kind schaukelt, die Frequenz mit der das Kind durch die Ruhelage hindurch schaukelt, bleibt die gleiche.